\documentclass{thesis}


% 定理类环境宏包
\usepackage{amsthm}

% 插图
\usepackage{graphicx}

% 三线表
\usepackage{booktabs}

% 表注
\usepackage{threeparttable}

% 跨页表格
\usepackage{longtable}

% SI 量和单位
\usepackage{siunitx}

% 配置图片的默认目录
\graphicspath{{figures/}}

% 数学命令
\makeatletter

\makeatother
\newcommand\eu{{\symup{e}}}
\newcommand\iu{{\symup{i}}}

\usepackage[backend=bibtex]{biblatex}

% hyperref 宏包在最后调用
\usepackage{hyperref}

\title{南京农业大学本科学位论文}
{ \LaTeX 模板示例文档}{A Sample Document for \LaTeX-based NJAU Thesis Template}
\college{公共管理学院}
\major{数学与应用数学}
\class{数学 114}
\studentnumber{114514191}
\author{ \LaTeX }
\advisor{ \LaTeX }
\advisortitle{副教授}

\begin{document}

\makecover

\copyrightpage

\pagenumbering{Roman}%中英文摘要要求使用罗马数字
\input{chapters/Abstract.tex}
\pagenumbering{arabic}

\tableofcontents

\mainpart

\section{文献综述}
\fontsize{12pt}{0pt}
\par \LaTeX 是由Leslie Lamport自主研发的一款全新\TeX 格式。
故事发生在一个被称作「地球」的真实世界,在这里,你将扮演一位名为「创作者」的神秘角色,导引排版公式之力,在自由的编辑中邂逅性格各异、能力独特的同伴们,和他们一起,利用强大的排版引擎和丰富的宏包库,自由地构建你的学术论文
——同时,逐步发掘「排版」的真相。

\section{数学}
\subsection{数字和单位}

\begin{itemize}
    \item \unit{\mu}
    \item \unit{\eta}
    \item \unit{\lambda}
    \item \unit{\xi}
\end{itemize}

\subsection{数学符号和公式}

\begin{equation}
    \vec{F}=m\frac{d\vec{v}}{dt}+\vec{v}\frac{dm}{dt}
\end{equation}

\subsection{证明}
\begin{proposition}
    Suppose $f$ is integrable on $\mathbb{R}^d$. Then for every $\epsilon > 0$:
    \begin{enumerate}
        \renewcommand{\theenumi}{\roman{enumi}}
        \item There exists a set of finite measure $B$ (a ball, for example) such
              that
              \begin{equation}
                  \int_{B^c} |f| < \epsilon.
              \end{equation}
        \item There is a $\delta > 0$ such that
              \begin{equation}
                  \int_E |f| < \epsilon \qquad \text{whenever } m(E) < \delta.
              \end{equation}
    \end{enumerate}
\end{proposition}
\begin{theorem}
    Suppose $\{f_n\}$ is a sequence of measurable functions such that
    $f_n(x) \to f(x)$ a.e. $x$, as $n$ tends to infinity.
    If $|f_n(x)| \le g(x)$, where $g$ is integrable, then
    \begin{equation}
        \int |f_n - f| \to 0 \qquad \text{as } n \to \infty,
    \end{equation}
    and consequently
    \begin{equation}
        \int f_n \to \int f \qquad \text{as } n \to \infty.
    \end{equation}
\end{theorem}

\section{图片与引用}
Front identification at the surface is important for defining WRs and
FRs. Generally, front identification is based on the gradient of ground or
low-level meteorological elements. The inclining feature of fronts together with the hilly terrain in South China make objective front
identification quite difficult. Therefore, in this paper, the locations of
fronts are analyzed by forecasters who subjectively locate fronts using a
combination of temperature and wind observations in the surface and
925-hPa layers. Specifically, if a front is strong (clear) enough in the
surface analyses, we locate the fronts based on the hourly surface observations (mainly by significant temperature gradients). Otherwise, we
determine the fronts mainly based on the 925-hPa directional wind
shear (significant difference in wind direction) and temperature gradients\cite{WU2020104693}
\begin{figure}[htbp]
	\centering
	\includegraphics[width=0.6\textwidth]{1-s2.0-S0169809519302285-gr1.jpg}
	\caption{Distributions of precipitation and the circulation situation at (a, c) 0800 LST on 10 May 2014 and (b, d) 0800 LST on 18 May 2014: (a, b) 925-hPa wind field (each bar represents 4 m/s) and cumulative precipitation in the past 24 h (shaded, mm); (c, d) vertical circulation along 113°E, meridional wind (shaded, m/s) and potential pseudo-equivalent temperature (contoured every 3 K). The solid brown lines and dotted blue lines in (a, b) identify the positions of the shear lines and 113°E, respectively. The solid squares and triangles in (c, d) are the latitudinal position of the precipitation areas and the fronts, respectively.}
    \label{fig:1}
\end{figure}

\section{结论与展望}
\input{chapters/Perspectives.tex}

\acknowledgement
首先特别感谢iamty制作的thesis-NJFU南京林业大学模板为本项目提供了宝贵的借鉴和参考,
同时Qsion在NJAU-Thesis项目中提供了历年理工科模板使我们在本次重制过程中少走了许多弯路。

此外,本项目学习了上海交通大学SJTUThesis,中国科学技术大学ustcthesis操作中使用的宏包拓展。

再次对诸多优秀的开源模板的贡献者表示衷心的感谢!

\thesisreferences
\end{document}